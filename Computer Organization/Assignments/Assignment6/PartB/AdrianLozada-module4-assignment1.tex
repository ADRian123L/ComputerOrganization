\documentclass[12pt]{article}
\usepackage[utf8]{inputenc}
\usepackage{amsmath}
\usepackage{amsfonts}
\usepackage{amssymb}
\usepackage{graphicx}
\usepackage[left=1in, right=1in, top=1in, bottom=1in]{geometry}
\usepackage{hyperref}
\usepackage{listings}
\usepackage{xcolor}
\usepackage{color}
\usepackage{minted}

\lstdefinelanguage{RISCV}{
  morekeywords={add, addi, addiw, addw, and, andi, auipc, beq, bge, bgeu, blt, bltu, bne, div, divu, divuw, divw, ecall, eret, fence, fence_i, jal, jalr, lb, lbu, ld, lh, lhu, lui, lw, lwu, mul, mulh, mulhsu, mulhu, mulw, or, ori, rem, remu, remuw, remw, ret, sb, sd, sh, sll, slli, slliw, sllw, slt, slti, sltiu, sltu, sra, srai, sraiw, sraw, srl, srli, srliw, srlw, sub, subw, sw, xor, xori},
  morecomment=[l]{\#},
  morecomment=[s]{/*}{*/},
  morekeywords=[2]{x0, x1, x2, x3, x4, x5, x6, x7, x8, x9, x10, x11, x12, x13, x14, x15, x16, x17, x18, x19, x20, x21, x22, x23, x24, x25, x26, x27, x28, x29, x30, x31},
  morekeywords=[3]{a0, a1, a2, a3, a4, a5, a6, a7},
  morekeywords=[4]{t0, t1, t2, t3, t4, t5, t6},
  morekeywords=[5]{s0, s1, s2, s3, s4, s5, s6, s7, s8, s9, s10, s11},
  keywordstyle=\color{cyan}, 
  keywordstyle=[2]\color{yellow},
  keywordstyle=[3]\color{orange},
  keywordstyle=[4]\color{red},
  keywordstyle=[5]\color{red},
}

\lstset{
  language=RISCV,
  basicstyle=\ttfamily\color{white},
  keywordstyle=\color{cyan},
  commentstyle=\color{green},
  numbers=left,
  numberstyle=\tiny\color{gray},
  frame=single,
  breaklines=true,
  breakatwhitespace=true,
  tabsize=2,
  backgroundcolor=\color[HTML]{1e1e1e}, % Dark background color
  rulecolor=\color{gray},
  numbersep=8pt,
}

\title{Assignment 6, Part A}
\author{Adrian Lozada}
\date{\today}

\begin{document}

\maketitle
\newpage

% Problem 1
    \section{Group}
    Group (Jiahui Dang)

% Problem 2
\section{Code Snippet}
    \begin{flushleft}
        The NAND instruction is not part of the RISC-V instruction set because the same
        functionality can be implemented using existing instructions. Write a short assembly code
        snippet that has the following functionality: s3 = s4 NAND s5. Use as few instructions as
        possible.
    \end{flushleft}
    \begin{lstlisting}[language=RISCV]
        and s3, s4, s5 # s3 = s4 & s5
        xori s3, s3, -1 # s3 = ~s3
    \end{lstlisting}
    
% Problem 3
    \section{Code Snippet 2}
    \begin{flushleft}
        Write RISC-V assembly code for placing the following immediate constants in
        register s7. Use a minimum number of instructions.
    \end{flushleft}
    % a:
    \text{a)}
    \begin{lstlisting}[language=RISCV]
        addi s7, x0, 45 # s7 = 45
    \end{lstlisting}
    % b:
    \text{b)}
    \begin{lstlisting}[language=RISCV]
        addi s7, x0, -199 # s7 = -199
    \end{lstlisting}
    % c:
    \text{c)}
    \begin{lstlisting}[language=RISCV]
        lui s7, 0xFEDCC # Load the upper 20 bits of the immediate into s7
        addi s7, s7, 0x8AB # Add the lower bits of the immediate to s7
    \end{lstlisting}
    % d:
    \text{d)}
    \begin{lstlisting}[language=RISCV]
        lui s7, 0xAABCD # Load the upper 20 bits of the immediate into s7
        addi s7, s7, 0x325 # Add the lower bits of the immediate to s7
    \end{lstlisting}

    \newpage
    % Problem 4
    \section{Code Snippet 3}
    \begin{flushleft}
        Convert the following high-level code into RISC-V assembly language. Assume
        that the signed integer variables g and h are in registers t0 and t1, respectively
    \end{flushleft}
    \text{a)}
    \begin{lstlisting}[language=RISCV]
    if:
        bge t1, t0, else # t1 >= t0
        addi t0, t0, 9 # t0 = t0 + 9
        slli t2, t0, 2 # t2 = t0 * 4
        srai t3, t0, 1 # t3 = t0 / 2
        add t0, t2, t3 # t0 = t0 * 5
        j end
    else:
        addi t1, t1, -5 # t1 = t1 - 5
        srai t1, t1, 3 # t1 = t1 / 8
    \end{lstlisting}

    \text{b)}
    \begin{lstlisting}[language=RISCV]
    if2:
        blt t1, t0, else2 # t1 < t0
        slli t2, t1, 1 # t2 = t1 * 2
        add t1, t2, t1 # t1 = t1 * 3
        add t0, t1, t0 # t0 = t0 + t1
        andi t0, t0, 15 # t0 = t0 & 15
        j end2 # jump to end
    else2:
        slli t0, t0, 1 # t0 = t0 * 2
        sub t0, t0, t1 # t0 = t0 - t1 
        slli t2, t1, 3 # t2 = t1 * 8
        slli t3, t1, 2 # t3 = t1 * 4
        add t1, t2, t3 # t1 = t1 * 12
    end2:
        \end{lstlisting}
    % Problem 5
    \newpage
    \section{Code Snippet 4}
    \begin{flushleft}
        Convert the following high-level code into RISC-V assembly language. Assume
        that the signed integer variables g and h are in registers t0 and t1, respectively. You can
        use other temporary registers like t2 and t3 if needed.        
    \end{flushleft}
    \text{a)}
    \begin{lstlisting}[language=RISCV]
        andi t2, t0, 15 # t2 = t0 % 16
        if:
            addi t3, x0, 9 # t3 = 9
            beq t2, t3, then # if t2 == t3
            andi t2, t2, 7 # t2 = t2 % 8
            addi t3, x0, 12 # t3 = 12
            beq t2, t3, then # if t2 == t3
            j else
        then:
            srai t1, t1, 1 # t1 = t1 / 2
            sub t1, t0, t1 # t1 = t0 - t1
            slli t2, t1, 1 # t2 = t1 * 2
            srai t3, t1, 1 # t3 = t1 / 2
            add t0, t2, t3 # t1 = t2 + t3
            add t0, t0, t1 # t0 = t0 + t1
            j end
        else:
            srai t2, t1, 1 # t2 = t1 / 2
            add t1, t1, t2 # t1 = t1 + t2
            add t1, t0, t1 # t1 = t1 + t0
            andi t0, t0, 7 # t0 = t0 % 8
        end:
    \end{lstlisting}

    \text{b)}
    \begin{lstlisting}[language=RISCV]
        andi t3, t0, 31 # t3 = t0 % 32
        if2:
            addi t2, x0, 6 # t2 = 6 
            bge t2, t3, else2 # if t2 >= t3
            addi t2, x0, 17 # t2 = 17
            bge t2, t3, else2  # if t2 >= t3
            slli t2, t1, 2 # t2 = t1 * 4
            slli t3, t1, 1 # t3 = t1 * 2
            add t1, t2, t3 # t1 = t1 * 7
            add t1, t0, t1 # t1 = t1 + t0
            andi t0, t0, 3 # t0 = t0 % 4
            j end2
        else2:
            slli t2, t1, 3 # t2 = t1 * 8
            srai t3, t1, 1 # t3 = t1 / 2
            sub t1, t2, t3 # t1 = t2 - t3
            add t1, t0, t1 # t1 = t1 + t0
            slli t2, t1, 3 # t2 = t1 * 8
            sub t1, t2, t1 # t1 = t2 - t1
        end2:
        
    \end{lstlisting}

    % Problem 6
    \newpage
    \section{Code Snippet 5}
    \begin{flushleft}
        Comment on each snippet with what the snippet does. Assume that there is an
        array, int arr [6] = \{3, 1, 4, 1, 5, 9\}, which starts at memory address
        0xBFFFFF00. You may assume each integer is stored in 4 bytes. Register a0 contains
        arr's address 0xBFFFFF00.
    \end{flushleft}
    \text{a)}
    \begin{lstlisting}[language=RISCV]
        # The code snippet takes the second and third element of the array 'arr' and adds them together.
        # It then stores the result in the first element of the array 'arr'.

        lw t0, 4(a0) # Load the value stored at the address 4(a0) into t0. This is the second element of the array.
        lw t1, 8(a0) # Load the value stored at the address 8(a0) into t1. This is the third element of the array.
        add t2, t0, t1 # Add t0 and t1 and store the result in t2
        sw t2, 0(a0) # Store the value at register t2 at the address 0(a0). This is the first element of the array.
    \end{lstlisting}

    \text{b)}
    \begin{lstlisting}[language=RISCV]
        # This code snippet implements a loop that iterates six times (the number of elements in the 'arr' array). 
        # During each iteration, it negates the value of the current element in the array.

            addi t0, a0, 0 # Initialize t0 to the start of the array
            addi t1, a0, 24 # Initialize t1 to the end of the array
        loop: beq t0, t1, end # If t0 == t1, end the loop
            lw t2, 0(t0) # Load the value stored at the address stores in t0 into t2
            sub t2, x0, t2 # Negate t2
            sw t4, 0(t0) # Store the value in t4 at the address 0(t0)
            addi t0, t0, 4 # Increment t0 by 4
            j loop # Jump to the loop label
        end:
    \end{lstlisting}

    % Problem 7
    \newpage
    \section{Code Snippet 6}
    \begin{flushleft}
        Write a RISC-V assembly snippet code to find the length of a string. Assume
        that the base address of string str is held in register a0. You may assume that each
        character is stored in 1 byte. You can use temporary registers if needed.
    \end{flushleft}
    \text{a)}
    \begin{lstlisting}[language=RISCV]
        # Function:
        # t0 = pointer to array
        # t1 = current character
        # t2 = length of string
        add t0, x0, a0 # t0 = a0
        while: 
            lbu t1, 0(t0) # t1 = *t0
            beq t1, x0, end # if t1 == 0, goto end
            addi t0, t0, 1 # t0++
            j while # goto while
        end: 
            sub t2, t0, a0 # t2 = t0 - a0
    \end{lstlisting}

\end{document}