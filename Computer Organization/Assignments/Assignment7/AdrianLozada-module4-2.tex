\documentclass[12pt]{article}
\usepackage[utf8]{inputenc}
\usepackage{amsmath}
\usepackage{amsfonts}
\usepackage{amssymb}
\usepackage{graphicx}
\usepackage[left=1in, right=1in, top=1in, bottom=1in]{geometry}
\usepackage{hyperref}
\usepackage{listings}
\usepackage{xcolor}
\usepackage{color}
\usepackage{minted}

\lstdefinelanguage{RISCV}{
  morekeywords={add, addi, addiw, addw, and, andi, auipc, beq, bge, bgeu, blt, bltu, bne, div, divu, divuw, divw, ecall, eret, fence, fence_i, jal, jalr, lb, lbu, ld, lh, lhu, lui, lw, lwu, mul, mulh, mulhsu, mulhu, mulw, or, ori, rem, remu, remuw, remw, ret, sb, sd, sh, sll, slli, slliw, sllw, slt, slti, sltiu, sltu, sra, srai, sraiw, sraw, srl, srli, srliw, srlw, sub, subw, sw, xor, xori},
  morecomment=[l]{\#},
  morecomment=[s]{/*}{*/},
  morekeywords=[2]{x0, x1, x2, x3, x4, x5, x6, x7, x8, x9, x10, x11, x12, x13, x14, x15, x16, x17, x18, x19, x20, x21, x22, x23, x24, x25, x26, x27, x28, x29, x30, x31},
  morekeywords=[3]{a0, a1, a2, a3, a4, a5, a6, a7},
  morekeywords=[4]{t0, t1, t2, t3, t4, t5, t6},
  morekeywords=[5]{s0, s1, s2, s3, s4, s5, s6, s7, s8, s9, s10, s11},
  keywordstyle=\color{cyan}, 
  keywordstyle=[2]\color{yellow},
  keywordstyle=[3]\color{orange},
  keywordstyle=[4]\color{red},
  keywordstyle=[5]\color{red},
}

\lstset{
  language=RISCV,
  basicstyle=\ttfamily\color{white},
  keywordstyle=\color{cyan},
  commentstyle=\color{green},
  numbers=left,
  numberstyle=\tiny\color{gray},
  frame=single,
  breaklines=true,
  breakatwhitespace=true,
  tabsize=2,
  backgroundcolor=\color[HTML]{1e1e1e}, % Dark background color
  rulecolor=\color{gray},
  numbersep=8pt,
}

\title{Assignment 7, Part A}
\author{Adrian Lozada}
\date{\today}

\begin{document}

\maketitle
\newpage

% Problem 1
    \section{}
    Independent

% Problem 2
\section{}
    \begin{flushleft}
        Write a RISC-V assembly function to search a specified integer in an integer
        array. The function should take the base address of the array, the number of elements in
        the array, and the specified integer as function arguments. The function should return the
        index number of the first array entry that holds the specified value. If no array element is the
        specified value, it should return the value -1.
    \end{flushleft}
    \begin{lstlisting}[language=RISCV]
# a0 is is the base of the array.
# a1 is the number of elements in the array.
# a2 is the number that the function searches for.

# The function returns the index in a0 (if found), otherwise -1.

find_int:
        addi t0, x0, 1 # t0 is 1.
        if: 
            blt a1, t0, return_1 # If size < 1 return -1
        add t0, a0, x0 # t0 is a pointer to a0.
        slli t3, a1, 2 # t3 is the size of the array in bytes.
        add t1 a0, t3 # t1 is a pointer to the last element of the array.
                
        loop:
            blt t1, t0, return_1 # t1 < t0
            lw t2, 0(t0) # t2 = *t0
            if_equal:
                    beq t2, a2, return_found # If t2 == a2 return the index.
            addi t0, t0, 4 # t0++
            jal x0, loop
                
        return_found:
                    sub a0, t0, a0 # Calculate the index.
                    srai a0, a0, 2
                    jalr x0, x1, 0
        return_1:
                addi a0, x0, -1 # Return -1.
                jalr x0, x1, 0
    \end{lstlisting}
    
% Problem 3
    \section{}
    \begin{flushleft}
        Consider a RISC-V assembly function func1. func1 has three passing
        arguments stored in registers a0, a1 and a2, uses temporary registers t0-t3 and
        saved registers s4-s10. func1 needs to call func2 and other functions may call func1
        also. func2 has two passing arguments stored in registers a0 and a1, respectively. In
        func1, after the program returns to func1 from func2, the code needs the original
        values stored in registers t1 and a0 before it calls func2.
    \end{flushleft}
    % a:
    \textbf{a.}
    \text{Ten words need to be stored in the stack.}
    \textbf{b.}
    \text{Funct1 needs to store the values inside a0, t1, s4-s10, ra in the stack.} 
    
    \newpage
    % Problem 4
    \section{}
    \begin{flushleft}
        Implement the C code snippet in RISC-V assembly language. Use s0-s2 to hold the variables
        i, j, and min_idx in the fucntion selectionSort. Be sure to hangle the stack pointer
        appropriatly. The array is stored on the stack of the selectionSort function. Clearly 
        comment your code.
    \end{flushleft}
    \text{a)}
    \begin{lstlisting}[language=RISCV]
        # selectionSort takes a0 (base address) and a1 (number of elements).
        selectionSort:
                        # s0 is i
                        # s1 is j
                        # s2 is min_idx
                        # s4 is a0
                        # s5 is a1
                        # s6 si a1 - 1 (size)
                        # s7 is min_idx

                        # Save the s registers into the stack:
                        addi sp, sp, -32
                        sw s0, 0(sp)
                        sw s1, 4(sp)
                        sw s2, 8(sp)
                        sw s3, 12(sp)
                        sw s4, 16(sp)
                        sw s5, 20(sp)
                        sw s6, 24(sp)
                        sw x1, 28(sp)

                        # Store a0 and a1:
                        add s4, a0, x0
                        add s5, a1, x0
            
                        # Sort the array:
                        addi s6, a1, -1 # n - 1 
                        add s0, x0, x0 # i = 0
                        for:
                            bge s0, s6, end_loop # i > n

                            # Call findMinimum function:
                            slli t0, s0, 2
                            add a0, s4, t0
                            sub a1, s5, s0
                            jal x1, findMinimum

                            # Set min_idx to the value in a0:
                            add s2, a0, x0

                            if_swap:
                                beq s2, s0, continue
                                # Call the swap function:
                                add t0, s2, s0
                                slli t0, t0, 2
                                add a0, s4, t0
                                slli t0, s0, 2
                                add a1, s4, t0
                                jal x1, swap

                            continue:

                            # Increment i:
                            addi s0, s0, 1

                            # Jump to Loop:
                            jal x0, for 

                        end_loop:

                        # Restore the s registers:
                        lw s0, 0(sp)
                        lw s1, 4(sp)
                        lw s2, 8(sp)
                        lw s3, 12(sp)
                        lw s4, 16(sp)
                        lw s5, 20(sp)
                        lw s6, 24(sp)
                        lw x1, 28(sp)

                        # Empty the stack:
                        addi sp, sp, 32

                        # Return:
                        jalr x0, x1, 0


        # findMinimum takes a0 (base address) and a1 (number of elements).
        findMinimum:
                    # t0 is min_idx
                    # t1 is min_E

                    # Initialize min_idx and min_E:
                    add t0, x0, x0
                    add t2, a0, t0
                    lw t1, 0(t2)

                    # Loop through the array:
                    # t3 is i
                    addi t3, x0, 1
                    for_loop_2:
                        # Condition:
                        bge t3, a1, end_loop_2
                        # Body:
                        slli t4, t3, 2
                        add t4, a0, t4
                        # t5 = arr[i]
                        lw t5, 0(t4)
                        if_2:
                            bge t5, t1, continue_2
                            add t0, t3, x0
                            slli t6, t0, 2
                            add t6, a0, t6
                            lw t1, 0(t6)

                        continue_2:
                        
                        # Increment i:
                        addi t3, t3, 1

                        # Jump to Loop:
                        jal x0, for_loop_2
                    
                    end_loop_2:
                    # Return min_idx:
                    add a0, t0, x0
                    # Return:
                    jalr x0, x1, 0

        # swap takes a0 (address of first element) and a1 (address of second element).
        swap:
                # t0 is temp
                # t1 is a0
		
                # Interchange the values:
                lw t0, 0(a0) # temp = a0
                lw t1, 0(a1) # a0 = a1 
                sw t1, 0(a0)
                sw t0, 0(a1)
	
                # Return:
                jalr x0, x1, 0
    \end{lstlisting}

    % Problem 5
    \newpage
    \section{}
    \begin{flushleft}
        Assume that the selectionSort is the function called. Draw the status of
        the stack before calling selectionSort and during each function call. Indicate stack
        addresses and names of registers and variables stored on the stack; mark the location
        of sp; and clearly mark each stack frame. Assume the sp starts at 0x8000.
    \end{flushleft}
    \text{a)}
    \begin{lstlisting}[language=RISCV]
    \end{lstlisting}

\end{document}